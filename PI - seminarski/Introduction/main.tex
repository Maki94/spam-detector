\section{Klasifikacija - klasifikacioni problem}

Klasifikacija predstavlja vrstu mašinskog učenja, koja je podoblast veštačke intaligencije čiji je cilj konstruisanje
algoritama i računarskih sistema koji su sposobni da se adaptiraju na nove
situacije i uče na osnovu iskustva. Razvijene su različite tehnike učenja za
izvršavanje različitih zadataka. Osnovne tehnike se tiču nadgledanog učenja za
diskreciono donošenje odluka, nadgledanog učenja za kontinuirano predviđanje i
pojačano učenje za sekvencionalno donošenje odluka, kao i nenadgledano učenje.

Većina praktičnih problema koristi oblik nadgedanog mašinskog učenja.
Ovaj model podrazumeva primenu nekog algoritma nad skupom ulaznih $X$ i
izlaznih promenljivih $Y$, trening podaci, za učenje mapiranja $Y \ = \ f(X)$.
Cilj je proceniti parametre funkcije $f$ tako da se ova funkcija može primeniti
za nove ulazne podatke $X$ za koje ne znamo izlaz $Y$, test podaci. Podela
nadgledanog učenja:
\begin{itemize}
  \item \textbf{Klasifikacija}: Problem identifikovanja kategorije klase novog
  posmatranja.
  \item Regresija: Problem predikcije kvantitivne vrednosti.
\end{itemize}

Nenadgledano učenje za ulazne podatke $X$ modelira strukture podataka ili
distribuciju podataka bez povratnih informacija $Y$. Cilj je
uočavanje zajedničkih svojstava podataka. Ovaj oblik učenja možemo svrstati:
\begin{itemize}
  \item Klasterizacija - metod za analizu grupisanja čiji je cilj
  particionisanje ulznih podataka na $k$ klastera.
  \item Asocijacija - metod za generaisanje pravila koja opisuju podatke.
\end{itemize}

Javlja se još jedan oblik mašinskog učenja, polu-nadgledano učenje. Labele su
dodeljene manjim brojem ulaznih podataka. Razlog može biti cena ručnog
procesiranja informacija.

Problem sa kojim se mi srećemo je upravo kvalitativne prirode, gde je potrbno
predvideti da li je komentar spam ili ne. Matematički ovu vrednost možemo
predstaviti kao binarnu vrednost (spam - 0, nije spam - 1).

Za klasifikaciju podataka se mogu koristiti klasifikatori kao što su logistička
regresija, linearna diskriminantna analiza, $k$ najbližih komšija, random forest,
stabla, support vector classifiers i drugi. Problem klasifikacije je složen i ne
postoji univerzalan klasifikator koji će raditi najbolje u svim situacijama.
